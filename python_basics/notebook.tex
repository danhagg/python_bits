
% Default to the notebook output style

    


% Inherit from the specified cell style.




    
\documentclass[11pt]{article}

    
    
    \usepackage[T1]{fontenc}
    % Nicer default font (+ math font) than Computer Modern for most use cases
    \usepackage{mathpazo}

    % Basic figure setup, for now with no caption control since it's done
    % automatically by Pandoc (which extracts ![](path) syntax from Markdown).
    \usepackage{graphicx}
    % We will generate all images so they have a width \maxwidth. This means
    % that they will get their normal width if they fit onto the page, but
    % are scaled down if they would overflow the margins.
    \makeatletter
    \def\maxwidth{\ifdim\Gin@nat@width>\linewidth\linewidth
    \else\Gin@nat@width\fi}
    \makeatother
    \let\Oldincludegraphics\includegraphics
    % Set max figure width to be 80% of text width, for now hardcoded.
    \renewcommand{\includegraphics}[1]{\Oldincludegraphics[width=.8\maxwidth]{#1}}
    % Ensure that by default, figures have no caption (until we provide a
    % proper Figure object with a Caption API and a way to capture that
    % in the conversion process - todo).
    \usepackage{caption}
    \DeclareCaptionLabelFormat{nolabel}{}
    \captionsetup{labelformat=nolabel}

    \usepackage{adjustbox} % Used to constrain images to a maximum size 
    \usepackage{xcolor} % Allow colors to be defined
    \usepackage{enumerate} % Needed for markdown enumerations to work
    \usepackage{geometry} % Used to adjust the document margins
    \usepackage{amsmath} % Equations
    \usepackage{amssymb} % Equations
    \usepackage{textcomp} % defines textquotesingle
    % Hack from http://tex.stackexchange.com/a/47451/13684:
    \AtBeginDocument{%
        \def\PYZsq{\textquotesingle}% Upright quotes in Pygmentized code
    }
    \usepackage{upquote} % Upright quotes for verbatim code
    \usepackage{eurosym} % defines \euro
    \usepackage[mathletters]{ucs} % Extended unicode (utf-8) support
    \usepackage[utf8x]{inputenc} % Allow utf-8 characters in the tex document
    \usepackage{fancyvrb} % verbatim replacement that allows latex
    \usepackage{grffile} % extends the file name processing of package graphics 
                         % to support a larger range 
    % The hyperref package gives us a pdf with properly built
    % internal navigation ('pdf bookmarks' for the table of contents,
    % internal cross-reference links, web links for URLs, etc.)
    \usepackage{hyperref}
    \usepackage{longtable} % longtable support required by pandoc >1.10
    \usepackage{booktabs}  % table support for pandoc > 1.12.2
    \usepackage[inline]{enumitem} % IRkernel/repr support (it uses the enumerate* environment)
    \usepackage[normalem]{ulem} % ulem is needed to support strikethroughs (\sout)
                                % normalem makes italics be italics, not underlines
    

    
    
    % Colors for the hyperref package
    \definecolor{urlcolor}{rgb}{0,.145,.698}
    \definecolor{linkcolor}{rgb}{.71,0.21,0.01}
    \definecolor{citecolor}{rgb}{.12,.54,.11}

    % ANSI colors
    \definecolor{ansi-black}{HTML}{3E424D}
    \definecolor{ansi-black-intense}{HTML}{282C36}
    \definecolor{ansi-red}{HTML}{E75C58}
    \definecolor{ansi-red-intense}{HTML}{B22B31}
    \definecolor{ansi-green}{HTML}{00A250}
    \definecolor{ansi-green-intense}{HTML}{007427}
    \definecolor{ansi-yellow}{HTML}{DDB62B}
    \definecolor{ansi-yellow-intense}{HTML}{B27D12}
    \definecolor{ansi-blue}{HTML}{208FFB}
    \definecolor{ansi-blue-intense}{HTML}{0065CA}
    \definecolor{ansi-magenta}{HTML}{D160C4}
    \definecolor{ansi-magenta-intense}{HTML}{A03196}
    \definecolor{ansi-cyan}{HTML}{60C6C8}
    \definecolor{ansi-cyan-intense}{HTML}{258F8F}
    \definecolor{ansi-white}{HTML}{C5C1B4}
    \definecolor{ansi-white-intense}{HTML}{A1A6B2}

    % commands and environments needed by pandoc snippets
    % extracted from the output of `pandoc -s`
    \providecommand{\tightlist}{%
      \setlength{\itemsep}{0pt}\setlength{\parskip}{0pt}}
    \DefineVerbatimEnvironment{Highlighting}{Verbatim}{commandchars=\\\{\}}
    % Add ',fontsize=\small' for more characters per line
    \newenvironment{Shaded}{}{}
    \newcommand{\KeywordTok}[1]{\textcolor[rgb]{0.00,0.44,0.13}{\textbf{{#1}}}}
    \newcommand{\DataTypeTok}[1]{\textcolor[rgb]{0.56,0.13,0.00}{{#1}}}
    \newcommand{\DecValTok}[1]{\textcolor[rgb]{0.25,0.63,0.44}{{#1}}}
    \newcommand{\BaseNTok}[1]{\textcolor[rgb]{0.25,0.63,0.44}{{#1}}}
    \newcommand{\FloatTok}[1]{\textcolor[rgb]{0.25,0.63,0.44}{{#1}}}
    \newcommand{\CharTok}[1]{\textcolor[rgb]{0.25,0.44,0.63}{{#1}}}
    \newcommand{\StringTok}[1]{\textcolor[rgb]{0.25,0.44,0.63}{{#1}}}
    \newcommand{\CommentTok}[1]{\textcolor[rgb]{0.38,0.63,0.69}{\textit{{#1}}}}
    \newcommand{\OtherTok}[1]{\textcolor[rgb]{0.00,0.44,0.13}{{#1}}}
    \newcommand{\AlertTok}[1]{\textcolor[rgb]{1.00,0.00,0.00}{\textbf{{#1}}}}
    \newcommand{\FunctionTok}[1]{\textcolor[rgb]{0.02,0.16,0.49}{{#1}}}
    \newcommand{\RegionMarkerTok}[1]{{#1}}
    \newcommand{\ErrorTok}[1]{\textcolor[rgb]{1.00,0.00,0.00}{\textbf{{#1}}}}
    \newcommand{\NormalTok}[1]{{#1}}
    
    % Additional commands for more recent versions of Pandoc
    \newcommand{\ConstantTok}[1]{\textcolor[rgb]{0.53,0.00,0.00}{{#1}}}
    \newcommand{\SpecialCharTok}[1]{\textcolor[rgb]{0.25,0.44,0.63}{{#1}}}
    \newcommand{\VerbatimStringTok}[1]{\textcolor[rgb]{0.25,0.44,0.63}{{#1}}}
    \newcommand{\SpecialStringTok}[1]{\textcolor[rgb]{0.73,0.40,0.53}{{#1}}}
    \newcommand{\ImportTok}[1]{{#1}}
    \newcommand{\DocumentationTok}[1]{\textcolor[rgb]{0.73,0.13,0.13}{\textit{{#1}}}}
    \newcommand{\AnnotationTok}[1]{\textcolor[rgb]{0.38,0.63,0.69}{\textbf{\textit{{#1}}}}}
    \newcommand{\CommentVarTok}[1]{\textcolor[rgb]{0.38,0.63,0.69}{\textbf{\textit{{#1}}}}}
    \newcommand{\VariableTok}[1]{\textcolor[rgb]{0.10,0.09,0.49}{{#1}}}
    \newcommand{\ControlFlowTok}[1]{\textcolor[rgb]{0.00,0.44,0.13}{\textbf{{#1}}}}
    \newcommand{\OperatorTok}[1]{\textcolor[rgb]{0.40,0.40,0.40}{{#1}}}
    \newcommand{\BuiltInTok}[1]{{#1}}
    \newcommand{\ExtensionTok}[1]{{#1}}
    \newcommand{\PreprocessorTok}[1]{\textcolor[rgb]{0.74,0.48,0.00}{{#1}}}
    \newcommand{\AttributeTok}[1]{\textcolor[rgb]{0.49,0.56,0.16}{{#1}}}
    \newcommand{\InformationTok}[1]{\textcolor[rgb]{0.38,0.63,0.69}{\textbf{\textit{{#1}}}}}
    \newcommand{\WarningTok}[1]{\textcolor[rgb]{0.38,0.63,0.69}{\textbf{\textit{{#1}}}}}
    
    
    % Define a nice break command that doesn't care if a line doesn't already
    % exist.
    \def\br{\hspace*{\fill} \\* }
    % Math Jax compatability definitions
    \def\gt{>}
    \def\lt{<}
    % Document parameters
    \title{0.Summary}
    
    
    

    % Pygments definitions
    
\makeatletter
\def\PY@reset{\let\PY@it=\relax \let\PY@bf=\relax%
    \let\PY@ul=\relax \let\PY@tc=\relax%
    \let\PY@bc=\relax \let\PY@ff=\relax}
\def\PY@tok#1{\csname PY@tok@#1\endcsname}
\def\PY@toks#1+{\ifx\relax#1\empty\else%
    \PY@tok{#1}\expandafter\PY@toks\fi}
\def\PY@do#1{\PY@bc{\PY@tc{\PY@ul{%
    \PY@it{\PY@bf{\PY@ff{#1}}}}}}}
\def\PY#1#2{\PY@reset\PY@toks#1+\relax+\PY@do{#2}}

\expandafter\def\csname PY@tok@w\endcsname{\def\PY@tc##1{\textcolor[rgb]{0.73,0.73,0.73}{##1}}}
\expandafter\def\csname PY@tok@c\endcsname{\let\PY@it=\textit\def\PY@tc##1{\textcolor[rgb]{0.25,0.50,0.50}{##1}}}
\expandafter\def\csname PY@tok@cp\endcsname{\def\PY@tc##1{\textcolor[rgb]{0.74,0.48,0.00}{##1}}}
\expandafter\def\csname PY@tok@k\endcsname{\let\PY@bf=\textbf\def\PY@tc##1{\textcolor[rgb]{0.00,0.50,0.00}{##1}}}
\expandafter\def\csname PY@tok@kp\endcsname{\def\PY@tc##1{\textcolor[rgb]{0.00,0.50,0.00}{##1}}}
\expandafter\def\csname PY@tok@kt\endcsname{\def\PY@tc##1{\textcolor[rgb]{0.69,0.00,0.25}{##1}}}
\expandafter\def\csname PY@tok@o\endcsname{\def\PY@tc##1{\textcolor[rgb]{0.40,0.40,0.40}{##1}}}
\expandafter\def\csname PY@tok@ow\endcsname{\let\PY@bf=\textbf\def\PY@tc##1{\textcolor[rgb]{0.67,0.13,1.00}{##1}}}
\expandafter\def\csname PY@tok@nb\endcsname{\def\PY@tc##1{\textcolor[rgb]{0.00,0.50,0.00}{##1}}}
\expandafter\def\csname PY@tok@nf\endcsname{\def\PY@tc##1{\textcolor[rgb]{0.00,0.00,1.00}{##1}}}
\expandafter\def\csname PY@tok@nc\endcsname{\let\PY@bf=\textbf\def\PY@tc##1{\textcolor[rgb]{0.00,0.00,1.00}{##1}}}
\expandafter\def\csname PY@tok@nn\endcsname{\let\PY@bf=\textbf\def\PY@tc##1{\textcolor[rgb]{0.00,0.00,1.00}{##1}}}
\expandafter\def\csname PY@tok@ne\endcsname{\let\PY@bf=\textbf\def\PY@tc##1{\textcolor[rgb]{0.82,0.25,0.23}{##1}}}
\expandafter\def\csname PY@tok@nv\endcsname{\def\PY@tc##1{\textcolor[rgb]{0.10,0.09,0.49}{##1}}}
\expandafter\def\csname PY@tok@no\endcsname{\def\PY@tc##1{\textcolor[rgb]{0.53,0.00,0.00}{##1}}}
\expandafter\def\csname PY@tok@nl\endcsname{\def\PY@tc##1{\textcolor[rgb]{0.63,0.63,0.00}{##1}}}
\expandafter\def\csname PY@tok@ni\endcsname{\let\PY@bf=\textbf\def\PY@tc##1{\textcolor[rgb]{0.60,0.60,0.60}{##1}}}
\expandafter\def\csname PY@tok@na\endcsname{\def\PY@tc##1{\textcolor[rgb]{0.49,0.56,0.16}{##1}}}
\expandafter\def\csname PY@tok@nt\endcsname{\let\PY@bf=\textbf\def\PY@tc##1{\textcolor[rgb]{0.00,0.50,0.00}{##1}}}
\expandafter\def\csname PY@tok@nd\endcsname{\def\PY@tc##1{\textcolor[rgb]{0.67,0.13,1.00}{##1}}}
\expandafter\def\csname PY@tok@s\endcsname{\def\PY@tc##1{\textcolor[rgb]{0.73,0.13,0.13}{##1}}}
\expandafter\def\csname PY@tok@sd\endcsname{\let\PY@it=\textit\def\PY@tc##1{\textcolor[rgb]{0.73,0.13,0.13}{##1}}}
\expandafter\def\csname PY@tok@si\endcsname{\let\PY@bf=\textbf\def\PY@tc##1{\textcolor[rgb]{0.73,0.40,0.53}{##1}}}
\expandafter\def\csname PY@tok@se\endcsname{\let\PY@bf=\textbf\def\PY@tc##1{\textcolor[rgb]{0.73,0.40,0.13}{##1}}}
\expandafter\def\csname PY@tok@sr\endcsname{\def\PY@tc##1{\textcolor[rgb]{0.73,0.40,0.53}{##1}}}
\expandafter\def\csname PY@tok@ss\endcsname{\def\PY@tc##1{\textcolor[rgb]{0.10,0.09,0.49}{##1}}}
\expandafter\def\csname PY@tok@sx\endcsname{\def\PY@tc##1{\textcolor[rgb]{0.00,0.50,0.00}{##1}}}
\expandafter\def\csname PY@tok@m\endcsname{\def\PY@tc##1{\textcolor[rgb]{0.40,0.40,0.40}{##1}}}
\expandafter\def\csname PY@tok@gh\endcsname{\let\PY@bf=\textbf\def\PY@tc##1{\textcolor[rgb]{0.00,0.00,0.50}{##1}}}
\expandafter\def\csname PY@tok@gu\endcsname{\let\PY@bf=\textbf\def\PY@tc##1{\textcolor[rgb]{0.50,0.00,0.50}{##1}}}
\expandafter\def\csname PY@tok@gd\endcsname{\def\PY@tc##1{\textcolor[rgb]{0.63,0.00,0.00}{##1}}}
\expandafter\def\csname PY@tok@gi\endcsname{\def\PY@tc##1{\textcolor[rgb]{0.00,0.63,0.00}{##1}}}
\expandafter\def\csname PY@tok@gr\endcsname{\def\PY@tc##1{\textcolor[rgb]{1.00,0.00,0.00}{##1}}}
\expandafter\def\csname PY@tok@ge\endcsname{\let\PY@it=\textit}
\expandafter\def\csname PY@tok@gs\endcsname{\let\PY@bf=\textbf}
\expandafter\def\csname PY@tok@gp\endcsname{\let\PY@bf=\textbf\def\PY@tc##1{\textcolor[rgb]{0.00,0.00,0.50}{##1}}}
\expandafter\def\csname PY@tok@go\endcsname{\def\PY@tc##1{\textcolor[rgb]{0.53,0.53,0.53}{##1}}}
\expandafter\def\csname PY@tok@gt\endcsname{\def\PY@tc##1{\textcolor[rgb]{0.00,0.27,0.87}{##1}}}
\expandafter\def\csname PY@tok@err\endcsname{\def\PY@bc##1{\setlength{\fboxsep}{0pt}\fcolorbox[rgb]{1.00,0.00,0.00}{1,1,1}{\strut ##1}}}
\expandafter\def\csname PY@tok@kc\endcsname{\let\PY@bf=\textbf\def\PY@tc##1{\textcolor[rgb]{0.00,0.50,0.00}{##1}}}
\expandafter\def\csname PY@tok@kd\endcsname{\let\PY@bf=\textbf\def\PY@tc##1{\textcolor[rgb]{0.00,0.50,0.00}{##1}}}
\expandafter\def\csname PY@tok@kn\endcsname{\let\PY@bf=\textbf\def\PY@tc##1{\textcolor[rgb]{0.00,0.50,0.00}{##1}}}
\expandafter\def\csname PY@tok@kr\endcsname{\let\PY@bf=\textbf\def\PY@tc##1{\textcolor[rgb]{0.00,0.50,0.00}{##1}}}
\expandafter\def\csname PY@tok@bp\endcsname{\def\PY@tc##1{\textcolor[rgb]{0.00,0.50,0.00}{##1}}}
\expandafter\def\csname PY@tok@fm\endcsname{\def\PY@tc##1{\textcolor[rgb]{0.00,0.00,1.00}{##1}}}
\expandafter\def\csname PY@tok@vc\endcsname{\def\PY@tc##1{\textcolor[rgb]{0.10,0.09,0.49}{##1}}}
\expandafter\def\csname PY@tok@vg\endcsname{\def\PY@tc##1{\textcolor[rgb]{0.10,0.09,0.49}{##1}}}
\expandafter\def\csname PY@tok@vi\endcsname{\def\PY@tc##1{\textcolor[rgb]{0.10,0.09,0.49}{##1}}}
\expandafter\def\csname PY@tok@vm\endcsname{\def\PY@tc##1{\textcolor[rgb]{0.10,0.09,0.49}{##1}}}
\expandafter\def\csname PY@tok@sa\endcsname{\def\PY@tc##1{\textcolor[rgb]{0.73,0.13,0.13}{##1}}}
\expandafter\def\csname PY@tok@sb\endcsname{\def\PY@tc##1{\textcolor[rgb]{0.73,0.13,0.13}{##1}}}
\expandafter\def\csname PY@tok@sc\endcsname{\def\PY@tc##1{\textcolor[rgb]{0.73,0.13,0.13}{##1}}}
\expandafter\def\csname PY@tok@dl\endcsname{\def\PY@tc##1{\textcolor[rgb]{0.73,0.13,0.13}{##1}}}
\expandafter\def\csname PY@tok@s2\endcsname{\def\PY@tc##1{\textcolor[rgb]{0.73,0.13,0.13}{##1}}}
\expandafter\def\csname PY@tok@sh\endcsname{\def\PY@tc##1{\textcolor[rgb]{0.73,0.13,0.13}{##1}}}
\expandafter\def\csname PY@tok@s1\endcsname{\def\PY@tc##1{\textcolor[rgb]{0.73,0.13,0.13}{##1}}}
\expandafter\def\csname PY@tok@mb\endcsname{\def\PY@tc##1{\textcolor[rgb]{0.40,0.40,0.40}{##1}}}
\expandafter\def\csname PY@tok@mf\endcsname{\def\PY@tc##1{\textcolor[rgb]{0.40,0.40,0.40}{##1}}}
\expandafter\def\csname PY@tok@mh\endcsname{\def\PY@tc##1{\textcolor[rgb]{0.40,0.40,0.40}{##1}}}
\expandafter\def\csname PY@tok@mi\endcsname{\def\PY@tc##1{\textcolor[rgb]{0.40,0.40,0.40}{##1}}}
\expandafter\def\csname PY@tok@il\endcsname{\def\PY@tc##1{\textcolor[rgb]{0.40,0.40,0.40}{##1}}}
\expandafter\def\csname PY@tok@mo\endcsname{\def\PY@tc##1{\textcolor[rgb]{0.40,0.40,0.40}{##1}}}
\expandafter\def\csname PY@tok@ch\endcsname{\let\PY@it=\textit\def\PY@tc##1{\textcolor[rgb]{0.25,0.50,0.50}{##1}}}
\expandafter\def\csname PY@tok@cm\endcsname{\let\PY@it=\textit\def\PY@tc##1{\textcolor[rgb]{0.25,0.50,0.50}{##1}}}
\expandafter\def\csname PY@tok@cpf\endcsname{\let\PY@it=\textit\def\PY@tc##1{\textcolor[rgb]{0.25,0.50,0.50}{##1}}}
\expandafter\def\csname PY@tok@c1\endcsname{\let\PY@it=\textit\def\PY@tc##1{\textcolor[rgb]{0.25,0.50,0.50}{##1}}}
\expandafter\def\csname PY@tok@cs\endcsname{\let\PY@it=\textit\def\PY@tc##1{\textcolor[rgb]{0.25,0.50,0.50}{##1}}}

\def\PYZbs{\char`\\}
\def\PYZus{\char`\_}
\def\PYZob{\char`\{}
\def\PYZcb{\char`\}}
\def\PYZca{\char`\^}
\def\PYZam{\char`\&}
\def\PYZlt{\char`\<}
\def\PYZgt{\char`\>}
\def\PYZsh{\char`\#}
\def\PYZpc{\char`\%}
\def\PYZdl{\char`\$}
\def\PYZhy{\char`\-}
\def\PYZsq{\char`\'}
\def\PYZdq{\char`\"}
\def\PYZti{\char`\~}
% for compatibility with earlier versions
\def\PYZat{@}
\def\PYZlb{[}
\def\PYZrb{]}
\makeatother


    % Exact colors from NB
    \definecolor{incolor}{rgb}{0.0, 0.0, 0.5}
    \definecolor{outcolor}{rgb}{0.545, 0.0, 0.0}



    
    % Prevent overflowing lines due to hard-to-break entities
    \sloppy 
    % Setup hyperref package
    \hypersetup{
      breaklinks=true,  % so long urls are correctly broken across lines
      colorlinks=true,
      urlcolor=urlcolor,
      linkcolor=linkcolor,
      citecolor=citecolor,
      }
    % Slightly bigger margins than the latex defaults
    
    \geometry{verbose,tmargin=1in,bmargin=1in,lmargin=1in,rmargin=1in}
    
    

    \begin{document}
    
    
    \maketitle
    
    

    
    \subsubsection{built-in functions}\label{built-in-functions}

\begin{Shaded}
\begin{Highlighting}[]
\BuiltInTok{bool}\NormalTok{([value])}
\BuiltInTok{dict}\NormalTok{(}\OperatorTok{**}\NormalTok{kwarg) }\OperatorTok{;}\NormalTok{ numbers }\OperatorTok{=} \BuiltInTok{dict}\NormalTok{(x}\OperatorTok{=}\DecValTok{5}\NormalTok{, y}\OperatorTok{=}\DecValTok{0}\NormalTok{)}
\BuiltInTok{dict}\NormalTok{(mapping, }\OperatorTok{**}\NormalTok{kwarg) }\OperatorTok{;}\NormalTok{ numbers2 }\OperatorTok{=} \BuiltInTok{dict}\NormalTok{(\{}\StringTok{'x'}\NormalTok{: }\DecValTok{4}\NormalTok{, }\StringTok{'y'}\NormalTok{: }\DecValTok{5}\NormalTok{\}, z}\OperatorTok{=}\DecValTok{8}\NormalTok{)}
\BuiltInTok{dict}\NormalTok{(iterable, }\OperatorTok{**}\NormalTok{kwarg) }\OperatorTok{;}\NormalTok{ numbers3 }\OperatorTok{=} \BuiltInTok{dict}\NormalTok{(}\BuiltInTok{dict}\NormalTok{(}\BuiltInTok{zip}\NormalTok{([}\StringTok{'x'}\NormalTok{, }\StringTok{'y'}\NormalTok{, }\StringTok{'z'}\NormalTok{], [}\DecValTok{1}\NormalTok{, }\DecValTok{2}\NormalTok{, }\DecValTok{3}\NormalTok{])))}
\BuiltInTok{dir}\NormalTok{([}\BuiltInTok{object}\NormalTok{]) }\CommentTok{# return a list of valid attributes of the object.}
\BuiltInTok{divmod}\NormalTok{(x, y) }\CommentTok{# returns a tuple (q, r)}
\BuiltInTok{enumerate}\NormalTok{(iterable, start}\OperatorTok{=}\DecValTok{0}\NormalTok{) }\OperatorTok{;} \BuiltInTok{enumerate}\NormalTok{(numbers3.values())}
\BuiltInTok{frozenset}\NormalTok{([iterable]) }\OperatorTok{;} \BuiltInTok{frozenset}\NormalTok{(numbers3.items()}
\BuiltInTok{iter}\NormalTok{(}\BuiltInTok{object}\NormalTok{[, sentinel]) }\CommentTok{# use as class method to make iterable objects}
\BuiltInTok{min}\OperatorTok{/}\BuiltInTok{max}\NormalTok{(arg1, arg2, }\OperatorTok{*}\NormalTok{args[, key]) }\OperatorTok{;} \BuiltInTok{max}\NormalTok{(iterable, key}\OperatorTok{=}\NormalTok{myFunc)}
\BuiltInTok{next}\NormalTok{(iterator, default) }\CommentTok{# use as class method to make iterable objects}
\BuiltInTok{pow}\NormalTok{(x, y[, z]) }\OperatorTok{;}\NormalTok{ x}\OperatorTok{**}\NormalTok{y }\OperatorTok{%}\NormalTok{ z}
\BuiltInTok{reversed}\NormalTok{(seq)}
\BuiltInTok{slice}\NormalTok{(start, stop, step) }\CommentTok{# sliceObj = slice(1,5,2); myList[sliceObj]}
\BuiltInTok{sorted}\NormalTok{(iterable[, key][, reverse])}
\BuiltInTok{sum}\NormalTok{(iterable, start) }\CommentTok{# sum will add the 'start' value if provided}
\BuiltInTok{zip}\NormalTok{(}\OperatorTok{*}\NormalTok{iterables) }\OperatorTok{;}
\end{Highlighting}
\end{Shaded}

\subsubsection{List Methods}\label{list-methods}

\begin{Shaded}
\begin{Highlighting}[]
\NormalTok{L.append(}\BuiltInTok{object}\NormalTok{) }\CommentTok{# append object to end}
\NormalTok{L.extend(iterable) }\CommentTok{# extend list by appending elements from the iterable}
\NormalTok{L.insert(index, }\BuiltInTok{object}\NormalTok{) }\CommentTok{# insert object before index}
\NormalTok{L.index(value, [start, [stop]]) }\CommentTok{# return first index of value.}
\NormalTok{L.count(value) }\CommentTok{# return number of occurrences of value}
\NormalTok{L.remove(value) }\CommentTok{# remove first occurrence of value.}
\NormalTok{L.pop([index]) }\CommentTok{# remove and return item at index (default last)}
\NormalTok{L.reverse() }\CommentTok{# reverse *IN PLACE*}
\NormalTok{L.sort(key}\OperatorTok{=}\VariableTok{None}\NormalTok{, reverse}\OperatorTok{=}\VariableTok{False}\NormalTok{)}
\end{Highlighting}
\end{Shaded}

\subsubsection{collections.deque({[}iterable{[},
maxlen{]}{]})}\label{collections.dequeiterable-maxlen}

\begin{Shaded}
\begin{Highlighting}[]
\NormalTok{dq.append(}\BuiltInTok{object}\NormalTok{) }\CommentTok{# append object to end of deque}
\NormalTok{dq.appendleft(}\BuiltInTok{object}\NormalTok{) }\CommentTok{# append object to beginning of deque}
\NormalTok{dq.extend(iterable) }\CommentTok{# extend deque by appending elements from the iterable}
\NormalTok{dq.extend(iterable) }\CommentTok{# extend deque by appending elements from the iterable}
\NormalTok{dq.insert(index, }\BuiltInTok{object}\NormalTok{) }\CommentTok{# insert object before index}
\NormalTok{dq.index(value, [start, [stop]]) }\CommentTok{# return first index of value.}
\NormalTok{dq.count(value) }\CommentTok{# return number of occurrences of value}
\NormalTok{dq.remove(value) }\CommentTok{# remove first occurrence of value.}
\NormalTok{dq.pop([index]) }\CommentTok{# remove and return item at index (default last)}
\NormalTok{dq.popleft([index]) }\CommentTok{# remove and return item at index (default first)}
\NormalTok{dq.reverse() }\CommentTok{# reverse *IN PLACE*}
\NormalTok{dq.rotate(n) }\CommentTok{# Rotate the deque n steps to the right (default n=1).  If n is negative, rotates left.}
\NormalTok{dq.sort(key}\OperatorTok{=}\VariableTok{None}\NormalTok{, reverse}\OperatorTok{=}\VariableTok{False}\NormalTok{)}
\end{Highlighting}
\end{Shaded}

\subsubsection{dict methods}\label{dict-methods}

\begin{Shaded}
\begin{Highlighting}[]
\NormalTok{myDict }\OperatorTok{=} \BuiltInTok{dict}\NormalTok{.fromkeys(iterable, value}\OperatorTok{=}\VariableTok{None}\NormalTok{, }\OperatorTok{/}\NormalTok{) }\CommentTok{# dict from a list all values=None}
\KeywordTok{del}\NormalTok{[myDict[key]] }\CommentTok{# delete item}
\NormalTok{myDict.update(y }\OperatorTok{=} \DecValTok{3}\NormalTok{, z }\OperatorTok{=} \DecValTok{0}\NormalTok{) }\CommentTok{# update, use a dict or tuple pair}
\NormalTok{myDict.popitem() }\CommentTok{# popitem() takes no argument}
\NormalTok{myDict.pop(key[, default])}
\NormalTok{myDict.get(key[, value]) }\CommentTok{# get the value or returns None or default}
\NormalTok{myDict.setdefault(}\StringTok{'knobber'}\NormalTok{, }\StringTok{'balls'}\NormalTok{) }\CommentTok{# dict.setdefault(key[, def_val])}
\end{Highlighting}
\end{Shaded}

\subsubsection{collections.defaultdict}\label{collections.defaultdict}

\begin{Shaded}
\begin{Highlighting}[]
\NormalTok{d }\OperatorTok{=}\NormalTok{ defaultdict(}\BuiltInTok{list}\NormalTok{)}
\ControlFlowTok{for}\NormalTok{ name }\KeywordTok{in}\NormalTok{ names:}
\NormalTok{    key }\OperatorTok{=} \BuiltInTok{len}\NormalTok{(name)}
\NormalTok{    d[key].append(name)}
    
\NormalTok{d }\OperatorTok{=}\NormalTok{ defaultdict(}\BuiltInTok{int}\NormalTok{)}
\ControlFlowTok{for}\NormalTok{ k }\KeywordTok{in}\NormalTok{ s:}
\NormalTok{    d[k] }\OperatorTok{+=} \DecValTok{1}
    
\NormalTok{d }\OperatorTok{=}\NormalTok{ defaultdict(}\BuiltInTok{set}\NormalTok{)}
\ControlFlowTok{for}\NormalTok{ k, v }\KeywordTok{in}\NormalTok{ s:}
\NormalTok{    d[k].add(v)}
\end{Highlighting}
\end{Shaded}

\subsubsection{set methods}\label{set-methods}

\begin{Shaded}
\begin{Highlighting}[]
\NormalTok{A.difference(B) }\CommentTok{# -}
\NormalTok{A.intersection(}\OperatorTok{*}\NormalTok{other_sets) }\CommentTok{# &}
\NormalTok{A.symmetric_difference(B) }\CommentTok{# ^}
\NormalTok{A.union(}\OperatorTok{*}\NormalTok{other_sets) }\CommentTok{# |}
\NormalTok{add(element)}
\NormalTok{discard(element) }\CommentTok{# Remove an element from a set if it is a member.}
\NormalTok{remove(element) }\CommentTok{# Remove an element from a set; it must be a member.}
\NormalTok{pop() }\CommentTok{# Remove and return an arbitrary set element.}
\end{Highlighting}
\end{Shaded}

\subsubsection{string methods}\label{string-methods}

\begin{Shaded}
\begin{Highlighting}[]
\NormalTok{S.endswith(suffix[, start[, end]]) }\CommentTok{# bool}
\NormalTok{S.count(sub[, start[, end]]) }\CommentTok{# Return the number of non-overlapping occurrences of sub in S}
\NormalTok{S.find(sub[, start[, end]]) }\CommentTok{# Return the low/high index in S where substring sub is found}
\NormalTok{S.ljust(width[, fillchar]) }\CommentTok{# Return S left-justified in a Unicode string of length width}
\NormalTok{S.join(iterable) }\CommentTok{# Return a concatenation of the strings in the iterable.  The separator between elements is S.}
\NormalTok{S.partition(sep) }\CommentTok{# (head, sep, tail)}
\NormalTok{S.replace(old, new[, count]) }\CommentTok{# all/count occurrences of substring old replaced by new.}
\NormalTok{S.split(sep}\OperatorTok{=}\VariableTok{None}\NormalTok{, maxsplit}\OperatorTok{=-}\DecValTok{1}\NormalTok{) }\CommentTok{# list of strings}
\NormalTok{S.splitlines([keepends]) }\CommentTok{# list of the lines in S, breaking at line boundaries unless keepends is given.}
\end{Highlighting}
\end{Shaded}

\subsubsection{collections.namedtuple(typename, field\_names, *,
verbose=False, rename=False,
module=None)}\label{collections.namedtupletypename-field_names-verbosefalse-renamefalse-modulenone}

\begin{Shaded}
\begin{Highlighting}[]
\ImportTok{from}\NormalTok{ collections }\ImportTok{import}\NormalTok{ namedtuple}

\NormalTok{ranks }\OperatorTok{=}\NormalTok{ [}\BuiltInTok{str}\NormalTok{(n) }\ControlFlowTok{for}\NormalTok{ n }\KeywordTok{in} \BuiltInTok{range}\NormalTok{(}\DecValTok{2}\NormalTok{,}\DecValTok{11}\NormalTok{)] }\OperatorTok{+} \BuiltInTok{list}\NormalTok{(}\StringTok{'JQKA'}\NormalTok{)}
\NormalTok{suits }\OperatorTok{=} \StringTok{'spades diamonds clubs hearts'}\NormalTok{.split()  }
\NormalTok{Card }\OperatorTok{=}\NormalTok{ namedtuple(}\StringTok{'Card'}\NormalTok{, [}\StringTok{'rank'}\NormalTok{, }\StringTok{'suit'}\NormalTok{])}
\NormalTok{b }\OperatorTok{=}\NormalTok{ Card(}\DecValTok{7}\NormalTok{, }\StringTok{'diamonds'}\NormalTok{)}

\NormalTok{Color }\OperatorTok{=}\NormalTok{ namedtuple(}\StringTok{'Color'}\NormalTok{, [}\StringTok{'red'}\NormalTok{, }\StringTok{'green'}\NormalTok{, }\StringTok{'blue'}\NormalTok{])}
\NormalTok{color }\OperatorTok{=}\NormalTok{ Color(}\DecValTok{50}\NormalTok{, }\DecValTok{160}\NormalTok{, }\DecValTok{220}\NormalTok{)}
\BuiltInTok{print}\NormalTok{(}\StringTok{"color.red"}\NormalTok{, color.red)}

\NormalTok{Point }\OperatorTok{=}\NormalTok{ namedtuple(}\StringTok{'Point'}\NormalTok{, [}\StringTok{'x'}\NormalTok{, }\StringTok{'y'}\NormalTok{])}
\NormalTok{t }\OperatorTok{=}\NormalTok{ [}\DecValTok{11}\NormalTok{, }\DecValTok{22}\NormalTok{]}
\BuiltInTok{print}\NormalTok{(}\StringTok{"t = [11, 22], Point._make(t) = "}\NormalTok{, Point._make(t))}

\NormalTok{p._make(iterable) }\CommentTok{# Make a new Point object from a sequence or iterable}
\NormalTok{p._asdict() }\CommentTok{# Return a new OrderedDict which maps field names to their values.}
\NormalTok{p._replace(}\OperatorTok{**}\NormalTok{kwds) }\CommentTok{# Return a new Point object replacing specified fields with new values}
\NormalTok{p._fields }\CommentTok{# Tuple of strings listing the field names. }
\end{Highlighting}
\end{Shaded}

\subsubsection{collections.Counter({[}iterable-or-mapping{]})}\label{collections.counteriterable-or-mapping}

\begin{Shaded}
\begin{Highlighting}[]
\NormalTok{c.elements() }\CommentTok{# Iterator over elements repeating each as many times as its count.}
\NormalTok{c.most_common(n}\OperatorTok{=}\VariableTok{None}\NormalTok{) }\CommentTok{# List the n most common elements and their counts.  If n is None-- list all.}
\NormalTok{c.keys() }\CommentTok{# a set-like object providing a view on D's keys}
\NormalTok{c.values() }\CommentTok{# an object providing a view on D's values}
\NormalTok{c.items() }\CommentTok{# a set-like object providing a view on D's items}
\BuiltInTok{set}\NormalTok{(c)}
\BuiltInTok{dict}\NormalTok{(c)}
\end{Highlighting}
\end{Shaded}

\subsubsection{itertools}\label{itertools}

\begin{Shaded}
\begin{Highlighting}[]
\NormalTok{accumulate(iterable[, func])}
\NormalTok{ombinations(iterable, r)}
\NormalTok{combinations_with_replacement(iterable, r)}
\NormalTok{permutations(iterable[, r])}
\NormalTok{compress(data, selectors)}
\NormalTok{count(start}\OperatorTok{=}\DecValTok{0}\NormalTok{, step}\OperatorTok{=}\DecValTok{1}\NormalTok{) }\CommentTok{# Return a count object whose .__next__() method returns consecutive values}
\NormalTok{cycle(iterable) }\CommentTok{# Return elements from the iterable. Then repeat indefinitely.}
\NormalTok{repeat(}\BuiltInTok{object}\NormalTok{ [,times]) }\CommentTok{# create an iterator which returns the object for # times}
\NormalTok{dropwhile(predicate, iterable) }\CommentTok{# Drop items from the iterable while predicate(item) is true. Afterwards, return every element until the iterable is exhausted.}
\NormalTok{takewhile(predicate, iterable) }\CommentTok{# Return successive entries from an iterable as long as the predicate evaluates to true for each entry.}
\NormalTok{zip_longest(iter1 [,iter2 [...]], [fillvalue}\OperatorTok{=}\VariableTok{None}\NormalTok{])}
\end{Highlighting}
\end{Shaded}


    % Add a bibliography block to the postdoc
    
    
    
    \end{document}
